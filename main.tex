\documentclass[a4paper,12pt,one]{article}
\usepackage[T2A]{fontenc} 
\usepackage[utf8]{inputenc} 
\usepackage[russian, english]{babel}
\usepackage{amsmath}
\usepackage{amsfonts}
\usepackage{amssymb}
\usepackage{comment}
\usepackage{graphicx}%Вставка картинок правильная
\usepackage{float}%"Плавающие" картинки
\usepackage{wrapfig}%Обтекание фигур
\usepackage{subcaption}
\usepackage{tipa}

\hyphenation{тур-бо-мо-ле-ку-ляр-ны-ми ана-ли-зи-ро-вать ме-тал-ли-чес-кой фор-ми-ру-ет-ся}
\parindent = 0pt
\addtolength{\hoffset}{-1cm}
\addtolength{\textwidth}{2cm}
\addtolength{\voffset}{-1.5cm}
\addtolength{\textheight}{1cm}
 
\title{MALDI-TOF}
\author{Ковалева Маргарита, Гребняк Ярослав }
\date{12 октября 2019}

\begin{document}

\maketitle

\section{Теоретическое введение}
Метод MALDI(Matrix Assisted Laser Desorption/Ionization) - метод ионизации, при котором вещество переводится из конденсированного состояния в газовую фазу под воздействием лазерного излучения. При таком подходе становится возможным <<поднять>> высокомолекулярные соединения; мяигкая ионизация позволяет анализировать их без фрагментации. Наиболее часто метод MALDI используется в сочетании с времяпролетным масс-анализатором(TOF = time of flight).

В методе MALDI облучению подвергается не вещество непосредственно, а его смесь с другим веществом, называемым матрицей. Анализируемое вещество, или аналит, и матрица расторяются в подходящем сольвенте а затем кристаллизуются на металлической подложке, называемой мишенью. В процессе сокристализации формируется кристаллическая решетка из молекул матрицы, в которую равномерно по всему объему вставлены молекулы аналита. 

После образования кристалла мишень помещается в прибор, система откачивается до давления порядка 0.75 торр,и мишень облучается лазером. В результате высокой плотности энергии матрица испаряется, нейтральные частицы удаляются турбомолекулярными насосами, а ионы переносятся во времяпролетный масс-анализатор.

\end{document}
